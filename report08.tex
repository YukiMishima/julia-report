\documentclass[12pt]{ltjsarticle}

\usepackage{amsmath}  %文字をイタリックにしない
\usepackage{graphicx}
\usepackage{cases}
\usepackage{tcolorbox}
\usepackage{listings}
\lstset{
% language={julia},                   %言語の指定
basicstyle={\ttfamily\small},        %書体の指定
backgroundcolor={\color[gray]{.95}}, %背景色と透過度
keywordstyle={\color{blue}},         %キーワード(int, ifなど)の書体指定
commentstyle={\color{OliveGreen}},   %注釈の書体 
stringstyle=\color{magenta},         %文字列
frame=single,                        %枠縁(leftline,topline,bottomline,lines,trBL,shadowbox, single)
numbers=left,                        %行番号表示
numberstyle={\ttfamily\small},       %行番号の書体指定
breaklines=true,                     %折り返し(自動改行)
breakindent = 10pt,                  %自動改行後のインデント量(デフォルトでは20[pt])	
tabsize=2,                           %タブの大きさ
captionpos=t                         %キャプションの場所(t,b : "tb"ならば上下両方に記載)
}
% \renewcommand{\lstlistingname}{Code} % キャプション名の指定

\begin{document}
\begin{titlepage}
\title{Report No.08 \\ Lorenz方程式と数値計算}
\author{まるまる研 B4 \\ なまえ}
\date{2018/12/19}
\maketitle
%\thispagestyle{empty
% \tableofcontents

\end{titlepage}

\section{目的}
Lorenz方程式の数値計算を通して、Lorenz方程式の数学的性質とそれが数値計算に与える影響について考察する。

\section{Lorenz方程式について}
Lorenz方程式とは以下のような常微分方程式である。パラメータはザルツマンのパラメータ、$\sigma$=10,r=28,b=8/3を採用する。
\begin{eqnarray}
    \begin{cases}
        \frac{du}{dt} = \sigma(-u+v), & \\
        \frac{dv}{dt} = u(-w+r)-v, & \\
        \frac{dw}{dt} = uv-bw. &
    \end{cases}
    \label{newton}
\end{eqnarray}

この方程式を多段階法+予測子修正子法により求めた近似解は以下のようになる。
\begin{figure}[htbp]
    \begin{center}
    \includegraphics[width=7cm]{Lorenz.png}
    \caption{Lorenz方程式の近似解}
        \end{center}
\end{figure}

また、この軌跡がプロットされていく様子のアニメーションは別ファイルにあるrolenz.gifである。
このソースコードは以下のとおりである。
\begin{lstlisting}[basicstyle=\ttfamily\footnotesize, frame=single]
function f(U)   # ローレンツ方程式を定義
    u, v, w = U
    
    return [
        σ * (- u + v)
        u * (- w + r) - v
        u*v - b*w
    ]
    
end

const σ, r, b = 10, 28, 8/3  # パラメータを指定
Δt = 0.01

function make_init(U)  # Runge-Kuttaを使って3段法に必要な6つの値を返す関数
    V = copy(U) 
    sq_V = copy(U)
    sq_f = f(U)
    
    for n in 1:2
        V = RK(V)
        sq_V = hcat(sq_V, V)
        sq_f = hcat(sq_f, f(V))
    end
    
    return sq_V, sq_f
end

function LM(U, vF)  # 線形多段階法+予測子修正子法
    Up = U + Δt * ( vF * [23/12, -4/3, 5/12] )
    fp = f(Up)
    Up = U + Δt * ( hcat(fp, vF) * [3/8, 19/24, -5/24, 1/24] )
    fp = f(Up)
    return Up, hcat(fp, vF[:,1:2])
end

sq_U, sq_f = make_init(U0)  # 初期値を生成
t_sq = Δt * [1:3003]

U = sq_U[:,3]
vF = copy(sq_f)

for i in 1:3000
    U, vF = LM(U, vF)
    sq_U = hcat(sq_U, U)
end

using Plots
gr()
using Interact

X, Y, Z = sq_U[1,:], sq_U[2,:], sq_U[3,:] # X,Y,Z座標を設定
plot(X, Y, Z, color="white")

@gif for i in 2999 # 細かな軌跡を追加でプロットし、アニメーションを作成
    x = sq_U[1,i:i+1]
    y = sq_U[2,i:i+1]
    z = sq_U[3,i:i+1]
    plot!(x,y,z, color="red",leg=false)
end
    \end{lstlisting}

線形常微分方程式では、変数がいくつあろうと解は、振動するか、収束するか、発散するかのいずれかの結果になる。
しかしながら、非線形常微分方程式であるLorenz方程式では、このいずれにもならず、不規則に振動を
続ける。これをカオスと呼ぶ。

\subsection{カオスの特徴}
カオスの特徴の一つとして、「初期値鋭敏性」が挙げられる。“ある場所で蝶が羽ばたいたことが原因で
数週間後に底と遠く離れた場所で竜巻が起こるかもしれない”
ということを比喩したバタフライ効果と同じように、非常に小さな初期値の変化がその後の値を大きく
変えてしまうという性質である。初期値鋭敏性の定義は以下の通りである。\\
\begin{tcolorbox}
    区間$J$に対し写像$f:J \rightarrow J$が与えられており、$x$は$J$の点であるとする.\\
    いま任意の$\delta$に対して、
    \begin{equation*}
        |x-y| < \delta \ かつ \ |f^{[n]}(x) - f^{[n]}(y)| > \varepsilon
    \end{equation*}
    となるような$J$に含まれる数$y$, 正の整数$n$, $\varepsilon(>0)$が存在するとき, $f$は
    $x$で初期値鋭敏性を持つあるいは$x$に鋭敏に依存するという. $f$が$J$のすべての点xで初期値鋭敏性を
    もつとき、$f$は$J$で初期値鋭敏性を持つ, あるいは$J$で鋭敏な依存性を持つという.\cite{chaos}
\end{tcolorbox}

十分近接した2点から出発した2つの軌道が時間の経過とともに乖離する度合いを示す量がリャプノフ指数
である。\\
\begin{tcolorbox}
    $J$を閉区間とし、$f:J \rightarrow J$は$J$で連続微分可能であるとする. $J$の与えられた点$x$
    に対し, $\lambda(x)$を, その極限が存在するとき,
    \begin{equation*}
        \lambda(x) \equiv \lim_{x \to \infty} \frac{1}{n}\text{ln}\left(f^{[n]}\right)'(x)|
    \end{equation*}
    で定義する. このとき, $\lambda(x)$を点$x$における$f$におけるリャプノフ指数という.\\
    $\lambda(x)$が$x$に依存しないとき, $\lambda(x)$を$\lambda$と記し, $f$のリャプノフ指数という.\cite{chaos}
\end{tcolorbox}

もし、fが安定な周期軌道を持つのであれば、軌道は収束するため乖離は起こらない。
したがってリャプノフ指数は負になる。逆に、初期値に鋭敏に依存するならば軌道間の距離は広がっていくため
リャプノフ指数は正になる。

\section{Lorenz方程式と数値計算}
以上の話から、Lorenz方程式の数値計算において何らかのスキームを用いた際に発生する近似誤差が、
カオスの初期値鋭敏性からその後の挙動が大きく変えてしまうという問題点が考えられる。
これが、Lorenz方程式の数値計算における問題点であると考察した。
実際に、Lorenz方程式を線形多段階法+予測子修正子法により求めた近似解と
Runge-Kutta法により求めた近似解の様子を比較した。ソースコードは上記のものに引き続いて
以下の通りとした。
\begin{lstlisting}
U = copy(U0)  #初期値をコピーしてくる
sq_U_RK = copy(U)   #配列に初期値を入れておく

for i in 1:3002  #Runge-Kutta法により近似解を求める
    U = RK(U)
    sq_U_RK = hcat(sq_U_RK, U)
end
X, Y, Z = sq_U[1,:], sq_U[2,:], sq_U[3,:]
plot(X, Y, Z, color="red")

@gif for i in 1:3001  # Runge-Kutta法により求めた近似解をプロットする
    x = sq_U_RK[1,i:i+1]
    y = sq_U_RK[2,i:i+1]
    z = sq_U_RK[3,i:i+1]
    plot!(x,y,z, color="green", leg=false)
end
\end{lstlisting}
このソースコードによって生成したアニメーションがrolenz by RK.gifである。
赤線の多段階法+予測子修正子法により求めた解と、緑線のRunge-Kutta法により求めた解は
異なる軌道を描くことが確認できた。したがって、丸め誤差や近似誤差により真の解と全く
異なる軌道を描く可能性があることは、Lorenz方程式の数値計算において忘れてはならない性質
であると言えるだろう。

\begin{thebibliography}{1}
    \bibitem{chaos} デニー・グーリック著, 前田恵一、原山卓久訳,
    『カオスとの遭遇 -力学系への数学的アプローチ-』第2版 (産業図書株式会社,1995)
    % \bibitem{chaos2} カオス入門 -現象の解析と数理-, 長島弘幸、馬場良和著, 1992
\end{thebibliography}
\end{document}